\documentclass{article}

\begin{document}


\section{Introduction}

Matching is 

\section{Experiments}

We perform a number of experiments to characterize the stable matching compared to the optimal matching and nearest neighbor matching.

\subsection{How often is stable equally good or better than xxx}

Through a large number of simulations we apply the three types matchings to randomly generated data.
[PROPERTIES OF THE RANDOMLY GENERATED DATA].
We measure how often stable matching is better than or equal to optimal matching on a range
of measures as suggested by Rubin [[FIXME: find ref].
For each simulation we record, 

\begin{itemize}
  \item The mean distance from case to matched control. Optimal matching will also yield the minimum possible value for this metric and stable matching will therefore only yield this value when the matching coincides with the optimal matching. 
  \item The median distance from case to matched control. Similar to mean distance, but more resilient to outliers, i.e., a single poorly matched case is unlikely to affect the median.
  \item The standard deviation of the distances between case and control. The standard deviation reflects the degree of imbalance of assigned controls.
\end{itemize}

The results are reported in a Venn diagram. 

\section{conclusion}

\end{document}

